%%%%%%%%%% 页面大小设置
%% 设置页边距,上下左右边框大小
% 如果页眉需要显示章节标题之类的,至少要1.5cm,页脚1.2cm则可以显示(设成a4paper格式又白调了...)
% \usepackage[top=1.2cm, bottom=1.2cm, left=0.8cm, right=0.8cm]{geometry}
% 重新调整
% 打印版极窄边框
\usepackage[a4paper, top=0.4cm, bottom=1.2cm, left=0.8cm, right=0.8cm]{geometry}
% 阅读版正常窄边框
% \usepackage[a4paper, top=1.27cm, bottom=1.27cm, left=1.27cm, right=1.27cm]{geometry}

%%%%%%%%%% 修改目录标题名称
% 默认是 Contents 字符串
\renewcommand{\contentsname}{目录}

%%%%%%%%%% 使用 CJKmainfont 选项指定支持中文的字体,以及相关设置
% \setCJKmainfont 设置 CJK 主字体,也就是设置 \rmfamily 的 CJK 字体
% \setCJKsansfont 设置 CJK 无衬线的字体,也就是设置 \sffamily 的 CJK 字体 (一般不用设置)
% \setCJKmonofont 设置 CJK 的等宽字体,也就是设置 \ttfamily 的 CJK 字体 (一般不用设置)

%%%% windows下
% \usepackage{xeCJK}  % 导入字体包
% \usepackage{fontspec} 
% \setCJKmainfont{Microsoft YaHei} 
% \setCJKmonofont{Microsoft YaHei Mono}

%%%%  ubuntu下
\usepackage{xeCJK}  % 导入字体包
\usepackage{fontspec} 
% \setCJKmainfont{Noto Serif CJK SC} 
% \setCJKmonofont{NotoSansCJK}

% 中文默认字体:{Noto Serif CJK SC,粗体为对应的粗体,斜体为 某种楷体
%   注意: 代码中的注释,也会变为斜体,楷体会比之前字体占得宽点好像,长度可能超出
%    2022-09-05 暂时好像不支持dart的代码块
\setCJKmainfont{Noto Serif CJK SC}
      [
        % BoldFont   = {Noto Sans Mono CJK TC:style=Bold},
        ItalicFont = {AR PL UKai CN}
      ]

%%% 针对中文自动换行
\XeTeXlinebreaklocale "zh"

%%% 设置行间距 1.0 倍
\linespread{1.0}\selectfont

%%% 段落之间的距离
\setlength{\parskip}{3pt}   

%%% 字与字之间加入0pt至1pt的间距,确保左右对齐
\XeTeXlinebreakskip = 0pt plus 1pt

%%%%%%%%%% 设置页眉页脚
% 预定义plain 没有页眉,页脚是居中的页码;
\usepackage{fancyhdr}
\pagestyle{plain}

%%%%%%%%%% 解决 ! LaTeX Error: Too deeply nested (lists more than 6 levels deep) 问题
\usepackage{enumitem}
\setlistdepth{9}

\setlist[itemize,1]{label=$\bullet$}
\setlist[itemize,2]{label=$\bullet$}
\setlist[itemize,3]{label=$\bullet$}
\setlist[itemize,4]{label=$\bullet$}
\setlist[itemize,5]{label=$\bullet$}
\setlist[itemize,6]{label=$\bullet$}
\setlist[itemize,7]{label=$\bullet$}
\setlist[itemize,8]{label=$\bullet$}
\setlist[itemize,9]{label=$\bullet$}
\renewlist{itemize}{itemize}{9}

\setlist[enumerate,1]{label=$\arabic*.$}
\setlist[enumerate,2]{label=$\alph*.$}
\setlist[enumerate,3]{label=$\roman*.$}
\setlist[enumerate,4]{label=$\arabic*.$}
\setlist[enumerate,5]{label=$\alpha*$}
\setlist[enumerate,6]{label=$\roman*.$}
\setlist[enumerate,7]{label=$\arabic*.$}
\setlist[enumerate,8]{label=$\alph*.$}
\setlist[enumerate,9]{label=$\roman*.$}
\renewlist{enumerate}{enumerate}{9}

%%%%%%%%%% 设置超链接的颜色,(但是不生效,不知道为什么)
% 2022-09-02 在 https://github.com/jgm/pandoc/issues/8226 中找到原因
%   需要在md文件的头部yaml中设置 boxlinks: true ,再配合 hyperref 的配置,即可生效。 之前使用的变量 colorlinks 无效了
%   不能在这个heperref的配置里直接该colorlinks为boxlinks,是因为它没有这个配置项,且删除colorlinks之后,链接颜色不生效
%   所有超链接都可以变颜色,包括toc。默认好像是红色
\usepackage{xcolor}
\usepackage{color}

\usepackage{hyperref}
% 有需要可以先制定颜色
\definecolor{linkcolor}{rgb}{0.9,0,0}
\definecolor{citecolor}{rgb}{0,0.6,0}
\definecolor{urlcolor}{rgb}{0,0,1}

\hypersetup{
    colorlinks=true,
    linkcolor=blue,
    filecolor=blue,      
    urlcolor=blue,
    citecolor=cyan,
}
\urlstyle{same}

%%%%%%%%%% 给 inline code 加上背景色
\usepackage{fvextra}
% 代码块中的注释超过一行,也能自动换行了(之前是延伸显示,超过了就看不到了)
% 但正文中的单行``注释,超过一行也显示不全
\DefineVerbatimEnvironment{Highlighting}{Verbatim}{breaklines,commandchars=\\\{\}}

\definecolor{bgcolor}{HTML}{E0E0E0}
\let\oldtexttt\texttt
\newcommand{\code}[1]{\begingroup\setlength{\fboxsep}{1pt}\colorbox{bgcolor}{\oldtexttt{\hspace*{2pt}\vphantom{A}#1\hspace*{2pt}}}\endgroup}
\renewcommand{\texttt}[1]{\code{\oldtexttt{#1}}}

%%%%%%%%%% 添加文字水印
% https://mirror-hk.koddos.net/CTAN/macros/latex/contrib/draftwatermark/draftwatermark.pdf
% \usepackage{ctex, draftwatermark, everypage} % 用这个正文文字大小有问题
\usepackage{draftwatermark} % 默认也是每一页都加水印,默认45°角
\SetWatermarkText{David Su | callmedavidsu@gmail.com} %关键字
\SetWatermarkLightness{0.8} %关键字的亮度 the lightness from 0 to 1, default 0.8, 1就不显示文字了
\SetWatermarkScale{0.3} %关键字的大小 default 1.2